% !TeX spellcheck = en_US
\documentclass{scrartcl}

\usepackage[utf8]{inputenc}
\usepackage[T1]{fontenc}
\usepackage[english]{babel}

\frenchspacing

\usepackage{amsmath}
\usepackage{mathtools}
\usepackage{nicefrac}
\usepackage[group-four-digits]{siunitx}
\sisetup{exponent-product = \cdot}

\usepackage{microtype}
\usepackage{csquotes}
\usepackage{booktabs}

\usepackage[dvipsnames]{xcolor}

\usepackage{tikz}
\usetikzlibrary{calc, decorations.pathreplacing, positioning}

\usepackage{pgffor}
\usepackage{graphicx}
\graphicspath{{./images/}{../src/results/}}
\usepackage{standalone}

\usepackage{caption}
\usepackage{subcaption}

\usepackage[backend=biber, style=authoryear, sorting=nyt, maxcitenames=2]{biblatex}
\addbibresource{main.bib}
\setlength{\bibitemsep}{0.3em}

\usepackage[ocgcolorlinks=true, citecolor=Green, breaklinks=true]{hyperref}
\usepackage[nameinlink,noabbrev]{cleveref}

\usepackage[super]{nth}
\newcommand{\MONTH}{%
	\ifcase\the\month
	\or January% 1
	\or February% 2
	\or March% 3
	\or April% 4
	\or May% 5
	\or June% 6
	\or July% 7
	\or August% 8
	\or September% 9
	\or October% 10
	\or November% 11
	\or December% 12
	\fi}

\newcommand{\ILR}{I^{\mathrm{LR}}}
\newcommand{\IHR}{I^{\mathrm{HR}}}
\newcommand{\IOUT}{I^{\mathrm{out}}}

\newcommand{\doubleplus}{+\mkern-8mu+}

\DeclareMathOperator*{\argmax}{arg\,max}
\DeclareMathOperator*{\argmin}{arg\,min}

\title{Super Resolution using Adversarial Network}
\subtitle{Deep Vision Project}
\author{Jonas Müller}
\date{\nth{\day} \MONTH, \the\year}

\begin{document}
\maketitle
\tableofcontents

\section{Introduction}
\subsection{Problem}
Single image super resolution is the task to create a \enquote{realistic} higher resolution image \( \IHR \) of a given image \( \ILR \).
This task is not well defined, because if we only know \( \ILR \) then \( \IHR \) is not well defined, e.\,g. in \cref{fig:pulse_img} there are three high resolution images that scale down to the same low resolution image (also in \cref{fig:resize}).

To overcome this problem in praxis and generate trainings data, a function \( f\colon \IHR \mapsto \ILR \) is defined, and it is assumed that \( f(\IHR) \) is an image that would upscale to \( \IHR \).
Since the dimension of high resolution images is far greater than the dimension of low resolution images this function can not be injective.
So this task is more complex than solving \( \argmin_{\hat I} \lVert f(\hat I) - \ILR \rVert_2 \) (\cref{fig:resize} shows the result of this optimization).

The function \( f \) is usually defined as \( f(x) = (x * g) \downarrow_{s} \) where \( g \) is some blurring kernel, \( s \) is the scaling factor, and \( (x\downarrow_{s})_{i,j} := x_{si,sj} \) \parencite{survey_sr}.
In this project the used kernel is a Gaussian kernel with standard deviation \(\sigma=1\).
This process is visualized in \cref{fig:resize}.

\begin{figure}
\begin{center}
\begin{subfigure}[t]{.47\textwidth}
	\centering
	\includegraphics[width=0.87\linewidth]{pulse_img.pdf}
	\caption{Multiple natural high resolution images that map to the same low resolution image. Adopted from \textcite{pulse}.}
	\label{fig:pulse_img}
\end{subfigure}\hfill%
\begin{subfigure}[t]{.47\textwidth}
	\centering
	\includestandalone[width=0.9\linewidth]{standalone/resize_process}
	\caption{The process of transforming a high resolution image (top left) to a low resolution image (bottom right) using \(s=4\).
	And another high resolution image (bottom left) generated by \( \argmin_{\hat I} \lVert f(\hat I) - \ILR \rVert_2 \).}
	\label{fig:resize}
\end{subfigure}%
\caption{Effects of scaling and uniqueness}
\label{fig:scaling}
\end{center}
\end{figure}

\subsection{Dataset}

The used dataset is the Flickr-Faces-HQ dataset by \textcite{style_gan}.
It consists of \num{70000} images of faces under creative commons licenses crawled from \href{https://www.flickr.com}{flickr.com}.
The original images are of size \( \num{1024} \times \num{1024} \).
The high resolution images in this project are to \( \num{256} \times \num{256} \) downscaled versions of these images.
\Cref{fig:ffhq-teaser} shows an sample of these images.

This dataset was then split into \num{55000} training, \num{10000} validation and \num{5000} test images.
The validation data was used to monitor the progress during training and to adjust the hyperparameters.
And the test set was first used when models were fully trained to compare the final performance of different approaches.

\begin{figure}
	\centering
	\includegraphics[width=.7\linewidth]{ffhq-teaser}
	\caption{Example images from the FFHQ dataset, graphic adopted from \textcite{style_gan}.}
	\label{fig:ffhq-teaser}
\end{figure} 

\section{Method}

\subsection{Model}
The model is split into four blocks. The whole model can be seen in \cref{fig:model}, the intention behind the individual blocks is described below.

The first block is composed of three convolutional layers with kernel size 1.
That is inspired by \textcite{color_net}, who found that the color space used as input affects the performance of the model.  
So this \enquote{Color Space Mapping}-block can be seen as a fully connected neural network on each pixel and is intended to learn a mapping to a favorable color space.
If the RGB color space is the most useful one for this problem, this block can easily learn the identity function.

The second block is a standard convolutional network.
It consists of four residual blocks each with two convolutional layers and an additional convolutional layer.
This block learns a abstract representation of the image which will then be used in the next block for upscaling.
The residual blocks here are sensible for two reasons: each residual block only needs to learn the changes necessary to the input but does not have to retain the input itself; the other reason it that the gradient can be carried backwards through many blocks.
The model could be improved by increasing the number of residual blocks.

The next block upscales the current feature using transposed convolutional layers.
The filters have a kernel size of \( 3 \) and stride \( 2 \), so each layer doubles the size, therefore \( \log_2(s) \) many layers (light green in \cref{fig:model}) are needed.
\textcite{sr_transposed} use this technique for upscaling in their model.
Another common way to increase the resolution of the image would be to use bicubic upscaling on \( \ILR \) and insert the upscaled image into the network, effectively learning to increase the sharpness of the image \parencite{srcnn_baseline, survey_sr}.
Using transposed convolutions brings the advantage, that the size of the perceptive field on \( \ILR \) is increased, because the upscaling expands the spaces between features of the image.
Also the amount of computations in the previous network is decreased because the image is smaller.
A disadvantage with this is that it only allows upscaling by factors of power two.
To solve this the kernel size and stride could be adjusted, e.\,g. for an upscaling factor of \( 3 \) a kernel size of \( 5 \) and stride \( 3 \) would give the desired result.
But this was not tested in this project.

The last block enables the network to only learn the difference between \( \IHR \) and the nearest neighbor upsample of \( \ILR \).
The basic idea here is to feed the input to the result so that the network can easily learn the identity function and then make small changes.
This is done often, e.\,g. by \textcite{vdsr}, usually by element-wise addition, i.\,e. \( T(\ILR) \oplus \mathrm{Up}(\ILR) \).
This block here concatenates the result of the upscaled version and the result of the network and again uses an pixel-wise fully connected network, to learn a more complex combination than addition.
Adding this block significantly decreased the time the model needed for training.


\begin{figure}
\begin{center}
	\includestandalone[width=\linewidth]{standalone/model}
	\caption{Each convolutional layer has a padding such that the size stays the same, the format is \enquote{Conv: \(k, f\)} where \(k\) is the kernel size and \(f\) the number of filters.
	The upscaling factor \(s\) has to be a power of two.
	The layers that are not specified have the same hyperparameters as the previous layer of the same type (type is indicated by color). And \( \doubleplus \) denotes the concatenation of both inputs.}
	\label{fig:model}
\end{center}
\end{figure}

\subsection{Perceptual Loss}

Perceptual loss was proposed by \textcite{perceptual_loss}.
The idea is to compare \( T(\ILR) \) and \( \IHR \) by their high level features and not by their pixel-wise difference.
For this an auxiliary network \( \phi \) is used, e.\,g. the VGG\textsubscript{16} network from \textcite{vgg} (see \cref{fig:vgg_perceptual}).
For a given layer \( \ell \), let \( \phi^\ell \) be the function that evaluates \( \phi \) up to \( \ell \).
We can then define the perceptual loss by
\[
	L_{\phi, \ell}(x, y) := \frac{1}{N} \lVert \phi^\ell(x)  - \phi^\ell(y) \rVert^2_2
	\,,
\]
where \( N \) is the number of outputs neurons of \( \ell \).

An idea that I wanted to try out was to use an discriminator network, that distinguishes upscaled from high resolution images, as the auxiliary network, a concept similar to a GAN \parencite{gan_goodfellow}.
The plan was that the discriminator would detect features that the model does not synthesize well and the model could then improve specifically these features.


\begin{figure}
\begin{center}
	\includestandalone[width=.8\linewidth]{standalone/vgg_perception}
	\caption{Part of the VGG\textsubscript{16} network.
		The proposed extraction layer for super resolution by \textcite{perceptual_loss} is \( \phi_{\mathrm{VGG_{16}}}^{2,2} \).}
	\label{fig:vgg_perceptual}
\end{center}
\end{figure}

\subsection{Regularization}\label{sec:reg}

When using a perceptual loss it is necessary to use regularization \parencite{perceptual_loss}.
Because optimizing through a convolutional layer will result in noisy images and regularization smooths the result, as was discussed on exercise sheet 9.

So by using regularization the result is pushed into the direction of the natural image manifold.
\textcite{srgan} proposes to train a discriminator, and use its loss as regularization to push the image in the direction of the manifold of trainings images.

\subsection{Discriminator}

The discriminator used in this project is a less deep variant of the discriminator used by \textcite{stylegan0}.
It is trained to distinguish upscaled images from original high resolution images.
For this six convolutional blocks followed by a shallow fully connected network are used, (\cref{fig:discriminator}).
Each convolutional block has two convolutional layers and a max pooling layer, so it halves the size of the input.

\begin{figure}
	\begin{center}
		\includestandalone[width=\linewidth]{standalone/disc}
		\caption{The discriminator}
		\label{fig:discriminator}
	\end{center}
\end{figure}

\section{Results}

\subsection{Evaluation Criteria}

Measuring the quality of upscaled images faces similar challenges as defining a loss function.
Therefore the most robust way for evaluation would be to asses the quality by humans, and in this report there is an emphasis on the visual results.
However there are also two commonly used metrics for automated evaluation, which will also be used here, because it is not viable to evaluate every image separately.

There is the peak signal-to-noise-ratio (PSNR) which is based on the mean squared error (MSE).
Because of this a higher PSNR does not necessarily imply a better visual quality.
It is defined (in dB) by
\[
	\mathrm{PSNR}(I^{\mathrm{orig}}, I^{\mathrm{approx}})
	= 10\cdot \log_{10}\Big(\frac{L^2}{\mathrm{MSE}(I^{\mathrm{orig}}, I^{\mathrm{approx}})}\Big)
	\,,
\]
where
\[
	L = \max I^{\mathrm{orig}}\,,
	\qquad
	\mathrm{MSE}(I^{\mathrm{orig}}, I^{\mathrm{approx}})
	= \frac{1}{3\cdot W \cdot H} \sum_{c=1}^3\sum_{n=1}^W\sum_{m=1}^H \Big(I^{\mathrm{orig}}_{cnm} - I^{\mathrm{approx}}_{cnm}\Big)^2
	\,.
\]

To improve on this problems the Structural Similarity Index Measure (SSIM) was designed:
\[
	\mathrm{SSIM}(I, J)
	= \frac{2\mu_I\mu_J + k_1}{\mu_I^2+\mu_J^2+k_1} \cdot \frac{\sigma_{IJ} + k_2}{\sigma_I^2 + \sigma_J^2 + k_2}
	\,,
\]
where \( \mu_I \) is the mean of \( I \), \( \sigma_{I^2} \) is the variance of \( I \), \( \sigma_{IJ} \) is the covariance of \( I \) and \( J \) and \( k_1, k_2 \) are constants.

For both criteria a higher score is a better score.

\subsection{Doubled Resolution}

The overall test results for doubled resolution are in \cref{tab:results_2x}.
But as mentioned before PSNR and SSIM alone are not sufficient to judge the visual quality.
So in \cref{fig:2x_upscaling_single}\footnote{Under \url{https://drive.google.com/drive/folders/1ZpuLoBAigyomHcMJQXsmpKrrJZJza-M3} 100 more upscaled images per model can be found, selected randomly from the test set.} there is an image from the test set upscaled by the different methods.
All models used ADAM as optimizer.

Bicubic interpolation is a static method that is included for reference.
It adds new pixels in between known pixels of the low resolution image by assuming a cubic function dependent on the nearest \num{16} pixel.
The resulting image has blurry edges and does not look sharper than the low resolution image, the advantage here it that is does not look as blocky as the original image.

Training the model with the MSE-loss gives the best result.
The model was trained with a learning rate between \num{e-4} and \num{e-3} and no regularization.
The image becomes as sharp and the color matches the original. 
This model achieves the highest score overall on the test set.

Training the model with perceptual loss using the VGG features does not give as good results.
It was trained with a learning rate between \num{e-4} and \num{e-3} and total variation regularization with strength of \num{e-4} as indicated by \textcite{perceptual_loss}.
The results look again sharp, but the colors do not match the original image.
This color mismatch results in scores lower than the bicubic interpolation but I would argue that the perceptual based model produces better results.

Lastly there is the model trained with perceptual loss and the discriminator as feature extractor.
This did not work out.
The general setup was as in \textcite{gan_goodfellow} alternating updates of the super resolution model and the discriminator.
The next problem was, that the model would adjust to quickly to the discriminator, causing the discriminator to be unable to learn the difference, even with really poor performance of the model.
This could be solving by using a learning rate of \num{e-4} for the discriminator and \num{5e-6} for the model.
With this setup the result stayed bad, but the discriminator was able to distinguish upscaled images.
The images were blurred and there was always a strong tint of different colors.
My hypothesis was that the discriminator overfitted to a single feature.
Next I added one-sided label smoothing \parencite{labelsmoothing}.
And after that noise to the input of the discriminator.
Both did not help.
The best result (used in \cref{fig:2x_upscaling_single} and \cref{tab:results_2x}) was obtained by using a model pretrained with the MSE-loss and a training time several times longer than for any other model after which the tint disappeared, but the blurry image remains.
During the attempt to get the training to work both \( \phi_{\mathrm{disc}}^{1,2} \) and \( \phi_{\mathrm{disc}}^{2,2} \) (\cref{fig:discriminator}) were used as feature extraction separately without noticeable improvement.

\begin{table}
	\centering
	\begin{tabular}{l SSSS}
		\toprule
		Method & {Bicubic} & {MSE} & {Perceptual (VGG)} & {Perceptual (Discriminator)} \\
		\midrule
		PSNR & 26.68 & 31.49 & 23.22 & 15.98 \\
		SSIM & 0.823 & 0.911 & 0.805 & 0.458 \\
		\bottomrule
	\end{tabular}
	\caption{Mean results of \( 2\times \) upscaling on the test set.}
	\label{tab:results_2x}
\end{table}


\begin{figure}
	\begin{center}
		\includestandalone[width=\linewidth]{standalone/2x_upscaling_single}
		\caption{Upscaling results of differently trained models with scaling factor of \num2.}
		\label{fig:2x_upscaling_single}
	\end{center}
\end{figure}

\bigskip
In \cref{table:non_faces} is an additional evaluation on Set 5 \parencite{set5} and Set 14 \parencite{set14}, which are often used as benchmark for super resolution and contain picture also non face images.
The models trained with MSE- and the perceptual-loss both achieve similar results to the result on faces.
So for this small scale factor the network seems to learn general rules that apply to different kind of pictures.

\begin{table}
\begin{center}
	\begin{tabular}{lSSSS}
		\toprule
		& {Set 5} & & {Set 14} & \\
		\cmidrule(lr){2-3} \cmidrule(lr){4-5}
		Model & {PSNR} & {SSIM} & {PSNR} & {SSIM} \\
		\midrule
		Bicubic & 27.30 & 0.857 & 25.23 & 0.777 \\
		MSE & 34.58 & 0.946 & 30.42 & 0.897 \\
		Perceptual (VGG) & 22.70 & 0.801 & 22.69 & 0.787 \\
		\bottomrule
	\end{tabular}
	\caption{Upscaling results on non-faces}
	\label{table:non_faces}
\end{center}
\end{table}

\subsection{Higher Scaling Factors}

Scaling pictures by a factor of \num{4} works well (\cref{fig:4x_upscaling_single}).
There are two methods used here:
First using the same model as before with two transposed convolutional layers and the same learning scheme as before.
Second, because the model only uses (transposed) convolutional and ReLU layers, the model can accept any input size, this can be used to chain a model previously trained on a lower scaling factor to achieve a higher factor, i.\,e. running the model from the previous section two times, will scale a image up by a factor of 4 (\enquote{MSE chained} in \cref{fig:4x_upscaling_single}).

Both approaches achieve better results than bicubic interpolation.
But the non-chained version is better than the chained version.
This is probably because it enables the network to add more complex features in an upscaling step that are not expressible by two simpler upscaling steps.

\begin{figure}
	\begin{center}
		\includestandalone[width=.8\linewidth]{standalone/4x_upscaling_single}
		\caption{Upscaling results of differently trained models  with scaling factor of \num4.}
		\label{fig:4x_upscaling_single}
	\end{center}
	\begin{center}
	\includestandalone[width=.8\linewidth]{standalone/8x_upscaling_single}
	\caption{Upscaling results of differently trained models  with scaling factor of \num8.}
	\label{fig:8x_upscaling_single}
\end{center}
\end{figure}

A scaling factor of \num{8} seems to be the end of the ability of my model (\cref{fig:8x_upscaling_single}).
For this upscaling scale I have again trained a model with MSE-loss without regularization, but also with a discriminator based regularization, as explained in \cref{sec:reg}.
Both approaches still give better results than bicubic interpolation, but the results are not as good as before.

A problem with this scale factor is that the upscale-skip connection in the model (dark green in \cref{fig:model}) uses nearest neighbor upsampling which results in now visible hard edges in the image.

\section{Conclusion and possible Improvements}

The model worked really well and produced some good result.
To improve the quality different types of upsampling for the skip-connection could be tried, e.\,g. bilinear or bicubic interpolation.
Also making the model deeper almost always improves on performance, this could be done by inserting more residual blocks or adding convolutional layers after each transposed convolutional upscaling.

Most problems of this project seem to come from the discriminator.
So the next step should probably be to implement a discriminator from a different paper, e.\,g. from \textcite{srgan} worked for super resolution.
But maybe the problem here stems from to little training time as GAN networks generally need much more time than I could give my network.

In the paper by \cite{perceptual_loss} they mention a post processing step, which matches the color distribution of the upscaled image to the input image.
I have not implemented this, which explains why my perceptual loss results are not as good as in the original paper.

All in all the once futuristic \enquote{Can you enhance this?} from crime shows is a lot more realistic today.


\clearpage
\nocite{*}
\printbibliography
\end{document}