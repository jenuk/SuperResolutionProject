\documentclass[margin=10mm]{standalone}

\usepackage[utf8]{inputenc}
\usepackage[T1]{fontenc}
\usepackage[english]{babel}

\usepackage{tikz}
\usetikzlibrary{calc, decorations.pathreplacing}

\begin{document}
	
\pgfdeclarelayer{bg}    % declare background layer
\pgfdeclarelayer{box}    % declare background layer
\pgfsetlayers{box,bg,main}  % set the order of the layers (main is the standard layer)
	
\resizebox{0.9\textwidth}{!}{
\begin{tikzpicture}[node distance=0.7cm]
	\tikzstyle{layer} = [draw, rectangle, minimum height=0.1cm, minimum width=3cm, rotate=90]
	\tikzstyle{conv} = [layer, fill=blue!20]
	\tikzstyle{transconv} = [layer, fill=green!20]
	\tikzstyle{relu} = [layer, fill=red!20]
	\tikzstyle{max} = [layer, fill=yellow!20]
	\tikzstyle{fc} = [layer, fill=magenta!30]
	\tikzstyle{data} = [line width=1mm, ->, color=blue]
	\tikzstyle{dataout} = [line width=1mm, ->, color=red]
	\tikzstyle{resdata} = [line width=0.5mm, ->, color=blue]
	\tikzstyle{resBlock} = [layer, fill=cyan!30]
	\tikzstyle{convBlock} = [layer, fill=teal!30]
	
	\node [layer] (input) {$I^{\mathrm{in}}$};
	
	\node [conv, below of=input, node distance=1.5cm] (conv11) {Conv: $3, 64$};
	\node [relu, below of=conv11] (conv11relu) {ReLU};
	\node [conv, below of=conv11relu] (conv12) {Conv: $3, 64$};
	\node [relu, below of=conv12] (conv12relu) {ReLU};
	\node [max, below of=conv12relu] (max1) {MaxPool: $2, 2$};
	
	\node [conv, below of=max1, node distance=1.5cm] (conv21) {Conv: $3, 128$};
	\node [relu, below of=conv21] (conv21relu) {ReLU};
	\node [conv, below of=conv21relu] (conv22) {Conv: $3, 128$};
	\node [relu, below of=conv22] (conv22relu) {ReLU};
	\node [max, below of=conv22relu] (max2) {MaxPool: $2, 2$};
	
	\node [layer, below of=max2, node distance=3cm, minimum height=3cm, align=center] (output) {Remaining \\ Network};
	
	\begin{pgfonlayer}{bg}    % select the background layer
		\draw [data] (input) -- (output);
		\draw [dataout] (conv22relu) -- ($(conv22relu) + (0, -3)$) node [below right, text=black] {$\phi_{\mathrm{VGG_{16}}}^{2,2}$};
	\end{pgfonlayer}
\end{tikzpicture}
}
\end{document}